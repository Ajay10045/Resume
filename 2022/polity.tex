\documentclass{article}

\begin{document}

\section*{Indian Constitution - Articles Overview}

\begin{center}
\begin{tabular}{|c|c|p{8cm}|c|}
\hline
\textbf{Article Range} & \textbf{Category} & \textbf{Description} & \textbf{Relevant Schedules} \\
\hline
1-4 & Preamble and Territory & Defines the territory of India and establishes the Union and its territories. & First Schedule \\
\hline
5-11 & Citizenship & Defines categories of citizenship and outlines the process of acquisition and termination. & - \\
\hline
12-35 & Fundamental Rights & Enumerates and protects fundamental rights such as equality, freedom, and protection against discrimination. & - \\
\hline
36-51 & Directive Principles & Outlines guidelines for state policy to achieve socio-economic justice. & - \\
\hline
52-78 & President & Specifies powers, election procedures, and functions of the President. & - \\
\hline
79-122 & Parliament & Details composition, powers, and functioning of Parliament, including Lok Sabha and Rajya Sabha. & - \\
\hline
123-147 & Union Judiciary & Defines the powers, independence, and functions of the Union Judiciary, including the Supreme Court. & - \\
\hline
148-151 & Miscellaneous Provisions & Covers topics related to the Comptroller and Auditor General, Election Commission, and Official Language. & Eighth Schedule, Ninth Schedule \\
\hline
152-162 & Governors & Defines the powers, appointment, and functions of Governors in the states. & - \\
\hline
163-171 & Special Provisions for Certain States & Enlists special provisions for states like Maharashtra and Gujarat. Recognizes unique historical and cultural contexts. & - \\
\hline
172-194 & State Legislatures & Details the composition, powers, and functioning of State Legislatures, accommodating variations between Legislative Assemblies and Legislative Councils. & - \\
\hline
195-212 & Financial Relations & Outlines the distribution of financial powers and responsibilities between the Union and the States. & Seventh Schedule \\
\hline
213-263 & Emergency Provisions & Outlines provisions for declaring a state of emergency due to war, external aggression, or armed rebellion. & - \\
\hline
264-300A & Distribution of Revenues & Details the distribution of revenues and taxes between the Union and the States. & - \\
\hline
301-307 & Trade, Commerce, and Intercourse & Ensures freedom of trade, commerce, and intercourse throughout India. & - \\
\hline
308-323 & Services Under the Union and States & Provides for services under the Union and the States, including the role of the Public Service Commissions. & - \\
\hline
323A, 323B & Tribunals & Empowers the establishment of administrative and other tribunals. & - \\
\hline
324-329A & Election Commission & Provides for the composition, powers, and functions of the Election Commission. & - \\
\hline
330-342 & Reservation of Seats & Ensures representation of Scheduled Castes, Scheduled Tribes, and Anglo-Indians in Parliament and State Legislatures. & - \\
\hline
343-351 & Official Language & Deals with the official language of the Republic of India. & Eighth Schedule \\
\hline
352-360 & Emergency Provisions & Outlines provisions for declaring a state of emergency due to war, external aggression, or armed rebellion. & - \\
\hline
361-367 & Special Provisions & Provides special protection and privileges for certain classes, including the President and Governors. & - \\
\hline
368 & Amendment of the Constitution & Specifies the procedure for amending the Constitution. & - \\
\hline
369-392 & Temporary and Transitional Provisions & Contains temporary and transitional provisions for the adaptation of existing laws. & - \\
\hline
393-395 & Short Title, Commencement, and Repeals & Deals with the short title, commencement, and repeal of the Constitution. & - \\
\hline
\end{tabular}
\end{center}

\end{document}
